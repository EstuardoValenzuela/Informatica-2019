
\documentclass{article}
\usepackage[utf8]{inputenc}

\usepackage[left=2cm,right=2cm,top=2cm,bottom=2cm]{geometry} 
\author{Estuardo Valenzuela(20181135) y Harold Marroqúin(20181193)} 
\title{Laboratorio No.3} 
\date{13 de Agosto 2018} 
\usepackage{natbib}
\usepackage{graphicx}
% Margins 
\topmargin= -0.45in 
\evensidemargin=0in
\oddsidemargin=0in
\textwidth=6.5in
\textheight=9.0in
\headsep=0.25in
\begin{document}
\linespread{1.1}

\maketitle

\section{Ejercicio 1}\par
\begin{center}

$[s(s(s(s(0))))] \oplus [s(s(s(0)))]$ 
\end{center}
\begin{center}
$[s(s(s(s(s(0)))))] \oplus [s(s(0))] $
\end{center}
\begin{center}
$[s(s(s(s(s(s(0))))))] \oplus [s(0)]$ 
\end{center}
\begin{center}
$[s(s(s(s(s(s(s(0)))))))] \oplus 0 $
\end{center}
\begin{center}
$[s(s(s(s(s(s(s(0\oplus 0)))))))]$
\end{center}
\begin{center}
$[s(s(s(s(s(s(s(0)))))))]$
\end{center}
\section{Ejercicio 2}
\[
n\otimes m := \left\{
\begin{array}{l l}
0 & \mbox{si } n=o \\
0 & \mbox{si } m=o \\
m & \mbox{si } n=1 \\
n & \mbox{si } m=1 \\ 
s(i)\otimes s(j) & \mbox{si } s(i) \oplus (s(i)\otimes j)\\
\end{array}
\right.
 \]
 \section{Ejercicio 3}
\begin{itemize}
    \item {$s(s(s(0))) \otimes 0$} 
\begin{flushleft}
      0  
\end{flushleft}
   \item{$s(s(s(0))) \otimes s(0)$}
\begin{flushleft}
    $s(s(s(0))) \otimes 1$
\end{flushleft}
\begin{flushleft}
    s(s(s(0)))
\end{flushleft}

    \item {$ s(s(s(0))) \otimes  s(s(0))$}
\begin{flushleft}
    $s(s(s(0))) \oplus [s(s(s(0))) \otimes s(0) ]$
\end{flushleft}
\begin{flushleft}
    $s(s(s(0))) \oplus (s(s(s(0))))$
\end{flushleft}
\begin{flushleft}
    $s(s(s(s(0)))) \oplus (s(s(0)))$
\end{flushleft}
\begin{flushleft}
    $s(s(s(s(s(0))))) \oplus (s(0))$
\end{flushleft}
\begin{flushleft}
   $s(s(s(s(s(s(0)))))) \oplus 0$
\end{flushleft}
\begin{flushleft}
    $s(s(s(s(s(s(0+0))))))$
\end{flushleft}
\begin{flushleft}
    $s(s(s(s(s(s(0))))))$
\end{flushleft}
\end{itemize}
\section{Ejercicio 4}
\begin{itemize}

        \item{$a\oplus s(s(0))=s(s(a))$} 
        \begin{flushleft}
        Caso base a=0 \\
        \end{flushleft}
        \begin{flushleft}
        $0 \oplus s(s(0)) = s(s(0))$ 
        \end{flushleft}
        \begin{flushleft}
        $s(s(0 \oplus 0)) =s(s(0))$ 
        \end{flushleft}
        \begin{flushleft}
        $s(s(0)) = s(s(0))$ 
        \end{flushleft}
        \begin{flushleft}
        Caso inductivo a= s(i) 
        \end{flushleft}
        \begin{flushleft}
        $s(i) \oplus s(s(0)) = s(s(s(i))) $ 
        \end{flushleft}
        \begin{flushleft}
        $s(s(i)) \oplus s(0) = s(s(s(i)))$ 
        \end{flushleft}
        \begin{flushleft}
        $s(s(s(i\oplus 0))) = s(s(s(i)))$ 
        \end{flushleft}
        \begin{flushleft}
        $s(s(s(i))) = s(s(s(i)))$
        \end{flushleft}
        \item{$a \otimes b = b \otimes a$}
        \begin{flushleft}
        Caso base a = 0 
        \end{flushleft}
        \begin{flushleft}
        $0 \otimes b = b \otimes 0$ 
        \end{flushleft}
        \begin{flushleft}
        $0 = 0$ 
        \end{flushleft}
        \begin{flushleft}
        Caso Inductivo a = s(i)
        \end{flushleft}
        \begin{flushleft}
            $s(i) \otimes b = b \otimes s(i)$ 
        \end{flushleft}
        \begin{flushleft}
            $s(i) \oplus (s(i)\otimes b) = (b \otimes s(i)) \oplus s(i)$
        \end{flushleft}
        \begin{flushleft}
            $s(i) \oplus (s(i)\otimes b) = s(i) \oplus (s(i) \otimes b)$
        \end{flushleft}

        \item{$a \otimes (b \otimes c)=(a\otimes b)\otimes c$}
        \begin{flushleft}
            Caso Base c= 0
        \end{flushleft}
        \begin{flushleft}
            $a \otimes (b\otimes 0) = (a \otimes b) \otimes 0$
        \end{flushleft}
        \begin{flushleft}
            $a \otimes 0  = (ab) \otimes 0$
        \end{flushleft}
        \begin{flushleft}
            $0=0$
        \end{flushleft}
        \begin{flushleft}
            Caso Inductivo a= s(i)
        \end{flushleft}
        \begin{flushleft}
            $s(i) \otimes (b \otimes c)=(s(i)\otimes b)\otimes c$
        \end{flushleft}
        \begin{flushleft}
            $s(i) \oplus (s(i) \otimes (b\otimes c)) = (s(i) \oplus (s(i) \otimes b)) \otimes c$
        \end{flushleft}
        \begin{flushleft}
           $s(i) \oplus (s(i) \otimes (b\otimes c)) = s(i) \oplus (s(i) \otimes (b\otimes c))$ 
        \end{flushleft}
        \item{$(a\oplus b)\otimes c = (a\otimes c) \oplus (b \otimes c)$}
        \begin{flushleft}
            Caso base c = 0
        \end{flushleft}
        \begin{flushleft}
            $(a\oplus b)\otimes 0 = (a\otimes 0) \oplus (b \otimes 0)$
        \end{flushleft}
        \begin{flushleft}
            $(a\oplus b)\otimes 0 = 0 \oplus 0$
        \end{flushleft}
        \begin{flushleft}
            $0 =0$
        \end{flushleft}
        \begin{flushleft}
            Caso Inductivo c= s(i) 
        \end{flushleft}
        \begin{flushleft}
            $(a\oplus b)\otimes s(i) = (a\otimes s(i)) \oplus (b \otimes s(i))$
        \end{flushleft}
        \begin{flushleft}
            $(a\otimes s(i)) \oplus (b \otimes s(i)) =  (a\otimes s(i)) \oplus (b \otimes s(i)) $
        \end{flushleft}
        \begin{flushleft}
            $(s(i) \oplus (s(i)\otimes a)) \oplus (s(i) \oplus (s(i)\otimes b)) = (s(i) \oplus (s(i)\otimes a)) \oplus (s(i) \oplus (s(i)\otimes b))$
        \end{flushleft}
\end{itemize}
\end{document}